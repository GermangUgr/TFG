
\chapter{Introducción}\label{ch:introduccion}

\begin{quotation}{\slshape
		An intelligent being cannot treat every object it sees as unique entity unlike anything else in the universe. It has to put objects in categories so that it may apply its hard-won knowledge about similar objects encountered in the past, to he object at hand.}
		\begin{flushright}
			\textbf{Steven Pinker, How the Mind Works, 1997} 
		\end{flushright}
\end{quotation}

Una de las habilidades más básicas y primitivas de la que están dotados los seres humanos es la de agrupar objetos similares para producir una clasificación que les resulte útil. Habilidad que ya nuestros más antiguos ancestros debieron poseer, por ejemplo, debieron ser capaces de darse cuenta de qué objetos eran comestibles, cuales eran venenosos y cuales intentarían matarles.

La capacidad de clasificación, en su sentido mas amplio, es necesaria para el desarrollo del lenguaje, que esta formado por palabras que nos ayudan a reconocer diferentes tipos de eventos, acciones y entidades. En esencia, cada sustantivo es una etiqueta que empleamos para agrupar un colectivo de seres u objetos con características similares, de manera que podemos hacer referencia a todos ellos empleando la palabra que los une.

De igual forma que la clasificación es una habilidad básica para las personas en su vida cotidiana, es también esencial en la mayoría de las ramas de la ciencia. En biología, por ejemplo, la clasificación de los diferentes tipos de organismos ha sido objeto de estudio desde el comienzo de su existencia. Aristóteles construyó un elaborado sistema de clasificación animal que dividía a todas las criaturas en dos grupos, aquellos con sangre roja y aquellos que carecían de ella. Más tarde propuso una subdivisión que los clasificaba según la forma en la que nuevos individuos venían al mundo, ya sea vivos, mediante huevos, crisálidas, etc.

Siguiendo a Aristóteles, Teoprastos escribió el primer documento que recopilaba las directrices para la clasificación de las plantas. Los trabajos resultantes fueron tan amplios y detallados que sentaron las bases para la investigación en biología durante los siguientes siglos. Este trabajo no fúe sustituido hasta 1737, cuando Carlos Linneo publicó su trabajo \textit{Genera Plantarum}, del que el siguiente fragmento ha sido extraído:

\begin{quotation}{\slshape
		All the real knowledge which we possess, depends on methods by which we distinguish the similar from the dissimilar. The greater the number of natural distinctions this method comprehends the clearer becomes our idea of things. The
		more numerous the objects which employ our attention the more difficult it becomes to
		form such a method and the more necessary.
		For we must not join in the same genus the horse and the swine, though both species
		had been one hoof’d nor separate in different genera the goat, the reindeer and the elk,
		tho’ they differ in the form of their horns. We ought therefore by attentive and diligent
		observation to determine the limits of the genera, since they cannot be determined a
		priori. This is the great work, the important labour, for should the genera be confused,
		all would be confusion.} 
		\begin{flushright}
			\textbf{Carlos Linneo, Genera Plantarum, 1737}
		\end{flushright}
\end{quotation}

La clasificación de los animales y las plantas ha sido de gran importancia en campos como la biología y la zoología. Particularmente, esta clasificación sentó las bases para el desarrollo de la teoría de la evolución de Darwin. Pero también ha sido de gran relevancia en áreas de conocimiento como la química y la física, con la clasificación de los elementos en la tabla periódica, propuesta por Mendeleyev en la década de 1860; o en astronomía, con la clasificación de estrellas en enanas o gigantes empleando las directrices de Hertzsprung–Russell.

\section{Motivación Personal}

Durante mi formación he escuchado de multitud de profesores que incorporar conocimiento humano a una máquina es una de las tareas más complejas a las que se ha enfrentado la humanidad.

Algo que realmente me resultó interesante fue que Deep Blue, la primera máquina en ganar al campeón del mundo de ajedrez, Gary Kasparov en aquel momento (1996), no ganó por un avance significativo en el algoritmo que ejecutaba la máquina, sino por avances en el hardware, que permitían que la máquina analizase más movimientos por unidad de tiempo, es decir, la máquina no empleaba conocimiento del que no disponía anteriormente, simplemente ``pensaba'' más rápido.

Por ello, he querido profundizar en el campo de la incorporación de conocimiento a las máquinas algo más de lo que he podido hacerlo durante estos años. El clustering con restricciones puede verse como un ejemplo de ello, al fin y al cabo no es más que guiar el proceso de toma de decisiones de una máquina incorporando conocimiento extraído de las personas.

\section{Objetivos}

%\let\cleardoublepage\relax
































%********************************************************************
% Appendix
\chapter{El algoritmo k-medias}\label{ap:kmeans}

El algoritmo \acf{KM} es el método básico para aplicar clustering, fue ideado por  Hugo Steinhaus en 1957, aunque no fue aplicado a un caso real hasta 1967, por James MacQueen \cite{KMEANS:1967}. 

El procedimiento que aplica K-medias es simple: asignar cada instancia al cluster cuyo centroide se encuentre mas cercano a ella, y calcular los centroides en base a las instancias asignadas al cluster asociado a cada uno. De esta manera, siguiendo la notación introducida en la sección \ref{ch:Algoritmos}, la regla de actualización de los centroides es:

\begin{equation}
v_i = \frac{1}{|c_i|} \sum_{x_i \in c_i} x_i
\label{a1}
\end{equation}

Así, si cada instancia se asigna al cluster correspondiente al centroide mas cercano, la función de error que minimiza K-medias es:

\begin{equation}
ERR = \frac{1}{2} \sum_{j = i}^{k} \sum_{x_i \in c_i} (v_j - x_i)^2
\label{a2}
\end{equation}

Formalizando la regla de asignación obtenemos:

\begin{equation}
c^i = argmin(||x_i - v_j||^2) \;\; t.q. \;\; j \in \{1, \cdots, k\}
\label{a3}
\end{equation}

donde $c^i$ representa el cluster seleccionado para la instancia $x_i$. Con todo, el proceso que K-medias sigue para obtener una partición de los datos queda resumido en el Algoritmo \ref{alg:kmedias}.

\begin{algorithm}
	
	\BlankLine
	\KwIn{Conjunto de datos $X$, numero de cluster resultantes $k$.}
	\KwOut{Partición $P$ del conjunto de datos $X$, centroides $V$.}
	\BlankLine
	\textbf{función} K-medias($X$, $k$) \Begin{
		\BlankLine
		1. Inicalización aleatoria de los centroides $V = \{v_1, \cdots, v_k\}$.\\
		2. Asignar cada instancia $x_i \in X$, al centroide más cercano $c_j$ siguiendo la ecuación \ref{a3}.\\
		3. Para cada cluster $c_i$, actualizar su centroide aplicando la ecuación \ref{a1}\\
		4. Iterar entre (2.) y (3.) hasta converger.\\
		5. \KwRet $C$, $V$
		\BlankLine
	}
	\BlankLine
	\caption{K-medias}
	\label{alg:kmedias}
\end{algorithm}


